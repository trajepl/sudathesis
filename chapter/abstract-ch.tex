\begin{abstract}

    % 所谓语义关联度计算,是指对于给定的一对评测对象,采用合适的方法,结合不同的背景知识,给出一个数值来表示两个对象在语义空间上的关联程度。这两个对象或者是文本中的词语,或者是两个商品,又或者是搜索关键词与文档,从评测对象来看,语义关联度计算是一项基础且十分重要的任务,在自然语言处理、推荐系统和计算机视觉方面都有相关的应用。其中词语之间的语义关联度计算吸引了很多研究者的兴趣。经典的计算方法主要利用了隐含在词典库或文本语料中的隐含语义关系来描述词语语义空间,取得了不错的效果。然而这些方法忽略了隐含在词语背后的语义网络,近年来,新提出的基于自由关联网络的方法改善了这个缺点,在语义关联度计算方面取得了更好的效果。但是这种方法需要事先对Wikipedia进行大量的复杂的预处理,此外,这种方法采用了固定的评分函数来衡量Wikipedia页面之间的相关性,这造成了模型灵活性和表达能力的欠缺。

    所谓语义关联度计算,是指对于给定的一组词语,用一个数值来表示两个词语在语义空间上的关联程度。此任务在自然语言处理和推荐系统中扮演着十分重要的角色,也吸引了很多研究者的兴趣。经典的计算方法主要利用了隐含在词典库或文本语料中的语义关系来描述词语语义空间,取得了不错的效果。然而这些方法忽略了词语所关联的语义网络,近年来新提出的基于自由关联网络的方法改善了这个缺点,取得了更好的效果。但是这种方法需要事先对文本语料进行复杂的预处理,且其采用的固定启发式函数造成了模型灵活性和表达能力的欠缺。

    为了改善传统自由关联网络方法中遇到的问题,本文根据词语与不同知识图谱实体之间的关系来建立知识关联网络,由此避免了复杂的文本预处理,同时本文采用了网络嵌入的方式来学习更加灵活鲁棒的知识表示,由此提升语义关联度计算的表现,具体来讲,本文的研究内容包含两个部分。

    (1)WordNet关联网络驱动的语义关联度计算。本文利用词语与WordNet实体间的关系构建WordNet关联网络。然后采用基于自注意力机制的网络嵌入模型来得到WordNet实体的词法向量表示。基于标准测评数据的实验结果表明,基于自注意力机制的网络嵌入模型可以更好得学习到实体的向量表示。此外,通过结合WordNet实体词法向量与文本语料上训练得到的词向量,可以更好地提升模型效果。

    (2)DBpedia关联网络驱动的语义关联度计算。本文基于TF-IDF思想建立起词语与DBpedia之间的关联,旨在利用实体之间的关系来丰富词语的语义信息。此外本文还提出了一种灵活的具有表达力的模型来学习词语背后的关联实体表征,这种模型同时将实体所处的属性空间与拓扑结构空间表示映射到向量空间,提高了模型的表达力。实验结果证明,此模型进一步提高了语义关联度计算的评测效果。

    相对于传统的自由关联网络,本文利用知识图谱构建的知识关联网络避免了复杂的文本预处理,同时本文针对基于不同知识库构建的知识关联网络给出了计算过程,相对于人为设计的启发式函数,本文提出的网络学习过程也更加灵活鲁棒,并取得了不错的效果。

\keywords{语义关联度,知识图谱,网络嵌入}

\end{abstract}

\vspace{-6.4pt}
\soochowauthor{李佳鹏~\quad}

\soochowtutor{赵~~~~雷\quad}
